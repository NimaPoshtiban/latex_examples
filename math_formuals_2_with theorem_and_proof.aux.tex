\documentclass{article}

\usepackage{amsmath,amssymb,amsthm,amsfonts}
\usepackage{xcolor} % using colors
\usepackage{soul}
\theoremstyle{remark} % --- customizing theorem style 
\newtheorem{theorem}{Theorem}[section] % --- defining a theorem with name "Theorem" the "section" optional argument add section numbering to theorems
\author{Nima Poshtiban}
\date{\today}
\title{Math Formulas Part 2}

\begin{document}
\maketitle
\section{\color{blue}Summation}
Here is the formula to calculate the total \setulcolor{red}\ul{Present} using Annual Uniform % draw a red underline under "Present".
\[
P_{0}=\sum_{n=1}^{m}{A\cdot\frac{1}{\left(1+i\right)^{n}}}
\]
\section{Integral and Limit}
Here is a famous math Integral
\[
\int_{}{}{x^{x} dx}
\]
Here is a simple area integral
\[
\iint_{}{x^{x}\cdot}{y^{y}}{dx dy}
\]
This is a famous integral
\[
\frac{1}{2\pi i} = \oint_{C}^{}{f(x)}
\]
Here is a limit function
%---- example of using a math font
\[
\mathbf{\lim_{x\to c}{f(x)} = \it{L}}
\]
\begin{theorem}[Akrra-Bazzi]
Akrra-Bazzi recursion problem\\
Here is the famous Akrra-Bazzi formula
\[
{T(n)} = \Theta \left(n^{p}\cdot\left(1+\int_{1}^{n}{\frac{f(x)}{x^{p+1}}dx}\right)\right)
\]
\end{theorem}
\begin{theorem}[Akrra-Bazzi]
	Akrra-Bazzi recursion problem\\
	Here is the famous Akrra-Bazzi formula
	\[
	{T(n)} = \Theta \left(n^{p}\cdot\left(1+\int_{1}^{n}{\frac{f(x)}{x^{p+1}}dx}\right)\right)
	\]
\end{theorem}
\begin{proof}
	I don't remember the proof lol !
\end{proof}

\section{Multi-Line Math formulas}
Here is a multi-line math equation
\begin{multline}
	y = \\2x + 2c +4f\\+ 4t + 9o \\- 2x + 2c +4f\\
\end{multline}
Here is how Feynman's Trick works
\begin{align}
		I &= \int_{0}^{1}{\ln\left(x\right)dx}&\\I(a)&= \int_{0}^{1}{\ln\left(ax\right)dx}\\I'(a) &= \frac{\partial a}{\partial x}\,\int_{0}^{1}{\ln\left(ax\right)\,dx}
\end{align}
Here is some famous functions
\begin{align}
	y &= x^{2} &y&=x^{3} \\ y&={1/x} &y&=\sqrt{x}
\end{align}

\section{Matrix}
\[
A =
\begin{vmatrix}
	1 & 2 \\
	3 & 4
\end{vmatrix}
\]
\[
A^{T} = 
\begin{vmatrix}
	1 & 3 \\
	2 & 4
\end{vmatrix}
\]

\section{Multi-Condition function}
\[
|x| = 
 \begin{cases}
 	x & \text{ for } x > 0\\
 	0 & \text{ for } x = 0\\
 	-x & \text{ for } x < 0
 \end{cases}
\]

%----- 

\end{document}